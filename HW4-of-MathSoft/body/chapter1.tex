\multicols{2}
\section{引言}
曼德博集合\cite{enwiki:1094796296}是一个在复平面上组成分形的点的集合,使用复二次多项式
进行迭代。曼德博集合的图像表现出一个精心设计的无限复杂的边界\cite{s1992fractal},
在放大倍率增加的情况下,它展示了越来越精细的递归细节;
在数学上,曼德博集合的边界被定义为是分形曲线,
此递归细节的样式取决于所检查的集合边界的区域。
\par 
曼德博集合在数学圈外也很受欢迎,
既因为它的美学吸引力,也因为它是应用简单规则产生的复杂结构
的一个例子。它是数学可视化,展现数学之美的最著名例子之一。
\par
本文将介绍并探究曼德博集合图像生成的原理以及通过计算机迭代
实现一定精度的曼德博集代码实现。

\section{背景介绍}
曼德博集合与分形定义\cite{mandelbrot1982fractal}紧密相关,而曼德博本人也提出了分形这个概念。
其中分形满足以下几个特点:
\begin{enumerate}
    \setlength{\itemsep}{-5pt}
    \item 具有无限精细的结构
    \item 比例自相似性
    \item 一般它的分数维大子它的拓扑维数
    \item 可以由简单的方法定义,并由递归、迭代产生
\end{enumerate}
\par 
而曼德博集合完美满足了上述几个特点,作为分形理论的典型例子,研究曼德博集合
具有重要的意义。

\section{数学理论}
利用复二次多项式来定义迭代:
$$f_c(z)=z^2+c$$
\par 
其中$c$是一个复数参数。从$z=0$开始迭代,$z_{n+1}=z^2_n+c$,每次迭代
的值依序如下数列所示:
$$(0,f_c(0),f_c(f_c(0)),f_c(f_c(f_c(0)))\cdots)$$
\par 
不同的参数$c$可能使迭代值的模逐渐发散到无限大,也可能收敛在有限的区域内。
曼德博集合$\mathrm{M}$就是使其不扩散的所有复数$c$的集合。
\par 
因为曼德博集合是闭集,再加上所有的点都包含围绕原点半径为2的封闭盘中,故
该集合是一个紧集。更具体地说,一个点$c$属于曼德博集合是闭集,当且仅当
$|z_n|\leq 2,\forall n\geq 0$.换句话说,曼德博集合中的点$c$必须满足
$z_n$的绝对值小于等于2,否则序列$z_n$将发散到无穷。


\section{算法}
\subsection{迭代算法}
由于需要计算机绘制曼德博集合的图像,并且使用无穷步迭代判断收敛,所以并不能
严格求解出每个点的收敛情况。因此,本文采用了一种近似算法。即设置一个最大迭代次数
\verb|maximum_iterator|,如果在迭代次数内$z_n>2$,则认为该点在迭代序列中会
发散指无穷,即不属于曼德博集合;反之则认为该点收敛,属于曼德博集合。同时,由于
在二维复平面当中点并不离散,所以将平面网格化,只考虑网格上的点是否属于曼德博集合,
并最终输出图像。
\par 
其中迭代算法的伪代码如下:
\begin{verbatim}
    For 图像中的每一个像素点 p
    c = p代表的复平面上的复数
    While 循环次数i小于maximum_iterator:
        If |z| > 2 then break
        Else z = z^2 + c
    If i = maximum_iterator then
        标记像素点为黑
    Else 标记像素点为白
\end{verbatim}

\subsection{绘制图像}
本文采用直接生成bmp格式的文件并改变其中的像素点RGB数值达到绘制图像的效果。
而bmp格式中每一个像素点由三个Byte构成,设置有关bmp格式的基本参数信息,然后
根据上述迭代算法得到每一个像素点的颜色信息,最后再将颜色信息存入对应的地址即可。

\section{数值算例}
通过上述算法思路实现的代码需要提供几个参数,分别是展示图像的中心点$ox$,$oy$以及全图
显示的图像半径尺寸$dimension$.同时,可以在程序中修改最大迭代次数$maximum_iterator$。
\par 
首先考虑改变最大迭代次数并输出图像,得到以下对应图像:
\begin{figure}[H]
    \centering
    \includegraphics[width=7cm]{./pic/graph1.bmp}
    \caption{N=10,ox=0,oy=0,dimension=2}
\end{figure}
\par 
\begin{figure}[H]
    \centering
    \includegraphics[width=7cm]{./pic/graph2.bmp}
    \caption{N=100,ox=0,oy=0,dimension=2}
\end{figure}
由此可见,迭代次数越大,图像也就更加精细,边界上的细节也就更多。
\par 
改变中心点的位置$ox$,$oy$,会得到以下对应图像:
\begin{figure}[H]
    \centering
    \includegraphics[width=7cm]{./pic/graph3.bmp}
    \caption{N=100,ox=0.3,oy=0.5,dimension=2}
\end{figure}
\par 
改变图像半径尺寸$dimension$,进行图像的缩放,得到一下对应图像。
\begin{figure}[H]
    \centering
    \includegraphics[width=7cm]{./pic/graph4.bmp}
    \caption{N=100,ox=0.3,oy=0.5,dimension=0.1}
\end{figure}
\par 
\begin{figure}[H]
    \centering
    \includegraphics[width=7cm]{./pic/graph5.bmp}
    \caption{N=100,ox=0.3,oy=0.45,dimension=0.01}
\end{figure}
不难发现,曼德博集合进行放大后仍然有着无限精细的结构,并且图案
具有某种程度上的相似性。由此可见,曼德博集合具有分形图形特性。

\section{结论}
由算法可知,曼德博集合图像可以通过简单的迭代得到;
同时,根据以上的数据算例,首先验证了算法的正确性,
其次不难发现,曼德博集合具有上下对称性。
其实不难从迭代公式本身推得该结论;同时,曼德博集合也满足分形的特征,
经过数倍放大之后仍具有无限精细的结构,并且图案具有某种程度的相似性。