\newpage
\section{问题确立}
课本在\textbf{P43}中的\textsc{Try It Out}中简述了如何在Shell中创建
一个简单的函数,代码如下:
\lstset{language=bash}
\begin{lstlisting}[caption = Try It Out]
#!/bin/sh
foo() {
    echo "Function foo is executing"
}
echo "script starting"
foo
echo "script ended"
exit 0
\end{lstlisting}
\par 
并且在后续的测试中发现,在传值进入新的函数中时,\textbf{\$* \$1 \$2 $\cdots$} 的值会改变,
直到函数执行完毕后再变回原来的值。但是课本中并没有讲到递归的实现情况。为此,我用Shell编写了一个
汉诺塔的程序。分析其中对应的\textbf{\$* \$1 \$2 $\cdots$}的值的变化情况。