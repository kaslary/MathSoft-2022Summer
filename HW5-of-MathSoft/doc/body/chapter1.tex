\newpage
\section{功能分析}
该代码计算了$f(x)=x^2-5$的根。
\par 
首先,代码设置了最大迭代次数$max_iter=100$.
\par 
其次,\verb|struct quadratic_params params = {1.0, 0.0, -5.0}|建立了
一个多项式$f(x)=x^2-5$。

\lstset{language=c}
\begin{lstlisting}[caption = roots.c]
    do{
        iter++;
        status = gsl_root_fsolver_iterate (s);
        r = gsl_root_fsolver_root (s);
        x_lo = gsl_root_fsolver_x_lower (s);
        x_hi = gsl_root_fsolver_x_upper (s);
        status = gsl_root_test_interval (x_lo, x_hi,
                                         0, 0.001);
        if (status == GSL_SUCCESS)
          printf ("Converged:\n");
        printf ("%5d [%.7f, %.7f] %.7f %+.7f %.7f\n",
                iter, x_lo, x_hi,
                r, r - r_expected, 
                x_hi - x_lo);
      }
    while (status == GSL_CONTINUE && iter < max_iter);
\end{lstlisting}

这一段代码利用了近似二分法的迭代求解根,其中\verb|gsl_root_fsolver_x_lower|返回了迭代的下界,
\verb|gsl_root_fsolver_x_upper|返回了迭代的上界,
\verb|gsl_root_fsolver_root|返回了迭代预计的根。不断的缩小迭代区间,直到最后
求解得到根或者迭代次数到达上限。
\par 
输出如下:
\begin{verbatim}
      using brent method
      iter [    lower,     upper]      root        err  err(est)
        1 [1.0000000, 5.0000000] 1.0000000 -1.2360680 4.0000000
        2 [1.0000000, 3.0000000] 3.0000000 +0.7639320 2.0000000
        3 [2.0000000, 3.0000000] 2.0000000 -0.2360680 1.0000000
        4 [2.2000000, 3.0000000] 2.2000000 -0.0360680 0.8000000
        5 [2.2000000, 2.2366300] 2.2366300 +0.0005621 0.0366300
    Converged:
        6 [2.2360634, 2.2366300] 2.2360634 -0.0000046 0.0005666
\end{verbatim}