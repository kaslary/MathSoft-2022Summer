\newpage
带有佩亚诺型余项的泰勒公式是解决多项式
的手段,而多项式逼近函数是近似计算和理论分析的一个
重要内容。
如果函数在\emph{f}在$x_0$处可导,则有
\begin{equation}
    f(x)=f(x_0)+f^{'}(x_0)(x-x_0)+o(x-x_0)
    \label{eq::eq1}
\end{equation}

即在点$x_0$附近,用一次多项式$f(x_0)+f^{'}(x_0)(x-x_0)$逼近函数$f(x)$时,其误差为$(x-x_0)$的高阶
无穷小量。下面考虑用二次或者高于二次的多项式去逼近。