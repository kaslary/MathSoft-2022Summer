\newpage
\section{系统环境}
通过 \verb|sudo lsb_release -a|指令\cite{kerrisk2010linux}可以查询系统版本号,结果如下: \begin{verbatim}
    Distributor ID: Ubuntu
    Description:    Ubuntu 21.04
    Release:        21.04
    Codename:       hirsute
\end{verbatim}

\section{主要配置}
\subsection{主要调整}
    \begin{enumerate}
        \item 更换了软件源
        \item 安装了中文输入法
        \item 添加安装了各种Python第三方库
    \end{enumerate}
\subsection{软件安装}
\begin{enumerate}
    \item Emacs
    \item Vscode
    \item Matlab
    \item Latex
    \item Git
    \item 魔法上网
\end{enumerate}
\par 
其中,重点讲述一下Emacs和Vscode的配置。
\par 
Emacs的调整主要包括新增实用快捷键(如调整多行代码顺序,快速多行注释等),显示行号,MiniBuffer
交互优化。后因为lisp管理较为麻烦,以及Vscode的可视化做的更加出色,故转为使用Vscode。
\par 
使用Vscode时安装了Emacs快捷键,以及对于C++,Python,以及Masm语言的扩展支持。同时安装了链接Latex的Latex Workshop,
以及Git管理相关的扩展插件,可视化Git操作。Vscode在可以编写代码的同时,也可以作为各种常见二进制文件的阅读器,基本做到了Coding All in One。


